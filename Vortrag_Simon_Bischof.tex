\documentclass{beamer}

\usepackage{sdq/templates/beamerthemekit}

\usepackage[utf8]{inputenc}
\usepackage[ngerman]{babel}
\usepackage[TS1,T1]{fontenc}
\usepackage{array}
\usepackage{xspace}
\usepackage{xcolor}
\usepackage{listings}
\usepackage[scaled]{beramono}
\usepackage{picins}

\newcommand{\N}{\mathbb{N}}

\author{Simon Bischof}
\title{Projektarbeit zur Vorlesung \glqq Graphalgorithmen und lineare Algebra Hand in Hand\grqq}
\subtitle{\glqq Dynamische Graphen generieren\grqq}
\date{17. Juli 2014}
\titleimage{images/title.png}

\begin{document}

\shorthandoff{"}

\begin{frame}
  \titlepage
\end{frame}

\begin{frame}
\frametitle{Kleine-Welt-Netzwerke}
\begin{itemize}
\item hohe Clusterung
\item geringe durchschnittliche Pfadlänge
\item Bildung von Hubs
\end{itemize}
\end{frame}

\begin{frame}
\frametitle{Watts-Strogatz}
\begin{enumerate}
\item Generieren eines regulären Ringgitters
\item Neuverbinden von Kanten mit bestimmter Wahrscheinlichkeit
\end{enumerate}
\end{frame}

\begin{frame}
\frametitle{Dorogotsev-Mendes}
\begin{enumerate}
\item Start mit einem Dreieck
\item Hinzufügen von Knoten:\pause
\begin{itemize}
\item Wähle zufällige Kante
\item Verbinde den Knoten mit den Kantenenden
\end{itemize}
\end{enumerate}
\end{frame}

\begin{frame}
\frametitle{Forest-Fire}
\begin{enumerate}
\item Start mit einem einzelnen Knoten
\item Hinzufügen von Knoten:\pause
\begin{itemize}
\item Wähle zufälligen Knoten
\item Simuliere ein von dort aus startendes Feuer\pause
\item Setze von einem Knoten eine Anzahl (zufällig nach geometrischer Verteilung) noch nicht brennender Knoten in Brand
\item Verbinde den Knoten mit allen brennenden Knoten
\end{itemize}
\end{enumerate}
\end{frame}
\end{document}